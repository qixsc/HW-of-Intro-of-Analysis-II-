\documentclass[12pt]{article}


\topmargin -40pt
\marginparwidth 0pt
 \oddsidemargin  -40pt
 \evensidemargin 0pt
 \marginparsep 0pt
\textwidth 7.2 in
 \textheight  10 in
 \hoffset  0.1in

\usepackage{amsthm,amsmath,amssymb,amscd,verbatim,epsfig}
\usepackage{mathptmx}
\usepackage{amsfonts}
%\usepackage{setapace}
\usepackage{graphicx}
\usepackage{bm}
%\usepackage{CJK}
\usepackage{ulem}
\usepackage{multicol}
\usepackage{enumerate}
\usepackage{float}
\usepackage{fontspec}
\usepackage{xeCJK}
\setmainfont{Times New Roman}
\setCJKmainfont{TaipeiSansTCBeta-Regular}
\XeTeXlinebreaklocale "zh"
\XeTeXlinebreakskip = 0pt plus 1pt




\title{Homework 1 of Introduction to Analysis(II)}
\author{AM15 黃琦翔 111652028}

\begin{document}
\maketitle
\begin{enumerate}
    \item \begin{enumerate}
        \item For any $x > 0$, and for any $\epsilon > 0$, we can find a $N \in \mathbb{N}$ s.t. $\epsilon x > \dfrac{1}{N}$.
        Also, we can get $x > \dfrac{1}{\epsilon N} > \dfrac{1}{N}$.
        Thus, $|g_k(x) - 0| = \dfrac{1}{kx} < \epsilon$ for all $k > N$.
        And for $x = 0$, whatever $k$ we take, $g_k(0) = n \cdot 0 = 0$.
        Therefore, for any $x > 0$, $\displaystyle\lim_{n\to\infty} g_n(x) = 0$.

        \item Assume for any $0 < \epsilon < 1$, and for a $N \in \mathbb{N}$,
        we have $|g_k(x) - 0| < \epsilon$ for all $x \geq 0$ with any $k \geq N$.
        Then, for $g_N$, we can find $x = \dfrac{1}{N}$ s.t. $g_N(x) = Nx = 1 > \epsilon$(contradiction).
        Thus, $g_n(x)$ is not uniform convergence on $x \geq 0$.

        For $x \geq c > 0$ and any $\epsilon > 0$, $|g_n(x) - 0| = \dfrac{1}{nx} < \dfrac{1}{nc} < \epsilon$ for some $n > N_c \in \mathbb{N}$.
        Thus, $g_n(x)$ is uniform convergence on $x \geq c > 0$.
    \end{enumerate}

    \item 
\end{enumerate}
\end{document}