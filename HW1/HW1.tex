\documentclass[12pt]{article}


\topmargin -40pt
\marginparwidth 0pt
 \oddsidemargin  -40pt
 \evensidemargin 0pt
 \marginparsep 0pt
\textwidth 7.2 in
 \textheight  10 in
 \hoffset  0.1in
\linespread{1.9}

\usepackage{amsthm,amsmath,amssymb,amscd,verbatim,epsfig}
\usepackage{mathptmx}
\usepackage{amsfonts}
%\usepackage{setapace}
\usepackage{graphicx}
\usepackage{bm}
%\usepackage{CJK}
\usepackage{ulem}
\usepackage{multicol}
\usepackage{enumerate}
\usepackage{float}
\usepackage{fontspec}
\usepackage{xeCJK}
\setmainfont{Times New Roman}
\setCJKmainfont{TaipeiSansTCBeta-Regular}
\XeTeXlinebreaklocale "zh"
\XeTeXlinebreakskip = 0pt plus 1pt




\title{Homework 1 of Introduction to Analysis(II)}
\author{AM15 黃琦翔 111652028}

\begin{document}
\maketitle
\begin{enumerate}
    \item \begin{enumerate}
        \item For any $x > 0$, and for any $\epsilon > 0$, we can find a $N \in \mathbb{N}$ s.t. $\epsilon x > \dfrac{1}{N}$.
        Also, we can get $x > \dfrac{1}{\epsilon N} > \dfrac{1}{N}$.
        Thus, $|g_k(x) - 0| = \dfrac{1}{kx} < \epsilon$ for all $k > N$.
        And for $x = 0$, whatever $k$ we take, $g_k(0) = n \cdot 0 = 0$.
        Therefore, for any $x > 0$, $\displaystyle\lim_{n\to\infty} g_n(x) = 0$.

        \item Assume for any $0 < \epsilon < 1$, exists $N \in \mathbb{N}$,
        we have $|g_k(x) - 0| < \epsilon$ for all $x \geq 0$ with any $k \geq N$.
        Then, for $g_N$, we can find $x = \dfrac{1}{N}$ s.t. $g_N(x) = Nx = 1 > \epsilon$(contradiction).
        Thus, $g_n(x)$ is not uniform convergence on $x \geq 0$.

        For $x \geq c > 0$ and any $\epsilon > 0$, there exists $N_c \in \mathbb{N}$ s.t. $N_c \cdot c > \dfrac{1}{\epsilon}$.
        Therefore, $|g_n(x) - 0| = \dfrac{1}{nx} < \dfrac{1}{nc} < \epsilon$ for all $n > N_c$.
        Thus, $g_n(x)$ is uniform convergence on $x \geq c > 0$.
    \end{enumerate}

    \item \begin{enumerate}
        \item $\ $ \begin{enumerate}
            \item[($\implies$)] Since $f_k \to f$ uniformly on $E$, for any $\epsilon > 0$, 
            exsits $N \in \mathbb{N}$ s.t. $d(f_k(x), f(x)) < \epsilon$ for all $x\in E$ and $k > N$.
            Thus, we can get for all $\epsilon > 0$, exists $N \in \mathbb{N}$ s.t. $\sup \{ d(f_k(x), f(x)) \mid x\in E\} < \epsilon$ for $k > N$.
            That means $\sup \{ d(f_k(x), f(x)) \mid x \in E\} \to 0$ as $k\to \infty$.
        
            \item[($\impliedby$)] Since $\sup \{ d(f_k(x), f(x)) \mid x\in E\} \to 0$ as $k \to \infty$, for any $\epsilon > 0$,
            exists $N \in \mathbb{N}$ s.t.
            
            $\sup \{ d(f_k(x), f(x)) \mid x \in E\} < \epsilon$ for $k > N$.
            That means $d(f_k(x), f(x)) < \epsilon$ for all $x$ and $k > N$.
            Therefore, $f_k \to f$ uniformly.
        \end{enumerate}
        \newpage

        \item $\ $\begin{enumerate}
            \item[($\implies$)] First, for any $\epsilon > 0$ and a $x_0 \in E$, we take$x_k \in \{ x \mid f_k(x) - f(x) > \epsilon, x_0\}$ for any $k$.
            Sicne $f_k \nrightarrow f$ uniformly, we can find some $k > N$ for any $N\in \mathbb{N}$ s.t. $f_k(x) - f(x) > \epsilon$.
            Thus, the sequence $\{ x_k\}$ satisfies that $\displaystyle\limsup_{k\to\infty} d(f_k(x_k), f(x_k)) \geq \epsilon > 0$.

            \item[($\impliedby$)] Since there exists a sequence $\{ x_k \}$ in $E$ s.t. $\displaystyle\limsup_{k\to\infty} d(f_k(x_k), f(x_k)) = \epsilon > 0$,

            $\sup \{ d(f_k(x), f(x)) \mid x\in E\} \nrightarrow 0$ as $k \to \infty$.
            Thus, by (a), we can get $f_k \nrightarrow f$.
        \end{enumerate}

        \item Let $f_k(x) = \dfrac{1}{k}e^{-k^2x^2}$ and $f(x) = 0$.
        First, we get $f_k'(x) = -2kxe^{-k^2x^2}$ and $f(x) > 0$ for all $x$.
        And it is positive for $x < 0$ and is negative for $x > 0$.
        Thus, the maximun of $f_k$ occurs at $x = 0$.
        
        Then, for any $\epsilon > 0$, there exists $N \in \mathbb{N}$ s.t. $N > \dfrac{1}{\epsilon}$.
        Therefore, $|f_k(x) - f(x)| < \dfrac{1}{k} < \dfrac{1}{N} < \epsilon$ for all $k > N$ and all $x\in \mathbb{N}$.
        Thus, $f_k \to f$ uniformly on $x \in \mathbb{R}$.

        For any $x \in \mathbb{R}$, and for any $\epsilon > 0$, exists $N \in \mathbb{N}$
        s.t. $2Nxe^{-N^2x^2} < \epsilon$ since 
        
        $\displaystyle\lim_{n\to\infty} ne^{-n^2x^2} = \displaystyle\lim_{n\to\infty} \dfrac{n}{e^{n^2x^2}} \overset{\rm L'H}{=} \displaystyle\lim_{n\to\infty} \dfrac{1}{2nx^2e^{n^2x^2}} = 0$.

        Thus, $|f_k'(x) - f'(x)| = |2kxe^{-k^2x^2} - 0| < |2Nxe^{-N^2x^2}| < \epsilon$ for all $k > N$.
        Therefore, $f_k' \to f'$ pointwisely.

        But for any interval contains $0$, we can find $N \in \mathbb{N}$ s.t. $(\dfrac{-1}{N}, 0]$ or $[0, \dfrac{1}{N})$ lies in the interval.
        Suppose $[0, \dfrac{1}{N})$ lies in the interval.
        Then, let $f_k''(x) = 2ke^{-2k^2x^2}(2k^2x^2 - 1) = 0$, we can get $x = \dfrac{1}{\sqrt{2} k}$.
        Then, for $\epsilon = \dfrac{1}{2}$, we can find $x = \dfrac{1}{\sqrt{2}k} \in (\dfrac{-1}{N}, \dfrac{1}{N})$ for all $k > \sqrt{2}N$.
        Thus, $|f_k'(x)| = 2\dfrac{1}{\sqrt{2}}e^{-\frac{1}{2}} = \sqrt{\dfrac{2}{e}} > \dfrac{1}{2}$.

        For the other case, we take $x = \dfrac{-1}{\sqrt{2}k}$ and the argument also right.
        Therefore, $f_k'(x) \nrightarrow f$ uniformly on any interval contains $0$.
    \end{enumerate}
\end{enumerate}
\end{document}