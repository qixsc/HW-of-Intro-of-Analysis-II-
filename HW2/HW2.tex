\documentclass[12pt]{article}

\topmargin -40pt
\marginparwidth 0pt
 \oddsidemargin  -40pt
 \evensidemargin 0pt
 \marginparsep 0pt
\textwidth 7.2 in
 \textheight  10 in
 \hoffset  0.1in
\linespread{1.9}

\usepackage{amsthm,amsmath,amssymb,amscd,verbatim,epsfig}
\usepackage{mathptmx}
\usepackage{amsfonts}
%\usepackage{setapace}
\usepackage{graphicx}
\usepackage{bm}
%\usepackage{CJK}
\usepackage{ulem}
\usepackage{multicol}
\usepackage{enumerate}
\usepackage{float}
\usepackage{fontspec}
\usepackage{xeCJK}
\setmainfont{Times New Roman}
\setCJKmainfont{TaipeiSansTCBeta-Regular}
\XeTeXlinebreaklocale "zh"
\XeTeXlinebreakskip = 0pt plus 1pt

\title{Homework 2 of Introduction to Analysis(II)}
\author{AM15 黃琦翔 111652028}

\begin{document}
\maketitle
\begin{enumerate}
    \item Suppose $f_k(x) = \displaystyle\sum_{n=1}^{k} \dfrac{x}{n^\alpha (1+nx^2)}$ and $E_I = [-L, L]$ for $L \in \mathbb{N}$.
    Then, we want to proof that 
    for all $\epsilon > 0$, there exists $N \in \mathbb{N}$ s.t. 
    $|f_k(x) - f_l(x)| < \epsilon$ for all $k, l > N$ and all $x \in I_L$.

    First, suppose that $l > k > N$, then \begin{align*}
        |f_k(x) - f_l(x)| &= \sum_{n=k}^l \dfrac{x}{n^\alpha (1+nx^2)}\\
        & \leq \sum_{n=k}^{l} \dfrac{L}{n^\alpha(1+nL^2)}\\
    \end{align*}


\end{enumerate}
\end{document}