\documentclass[12pt]{article}

\topmargin -40pt
\marginparwidth 0pt
 \oddsidemargin  -40pt
 \evensidemargin 0pt
 \marginparsep 0pt
\textwidth 7.2 in
 \textheight  10 in
 \hoffset  0.1in
\linespread{1.9}

\usepackage{amsthm,amsmath,amssymb,amscd,verbatim,epsfig}
\usepackage{mathptmx}
\usepackage{amsfonts}
%\usepackage{setapace}
\usepackage{graphicx}
\usepackage{bm}
%\usepackage{CJK}
\usepackage{ulem}
\usepackage{multicol}
\usepackage{enumerate}
\usepackage{float}
\usepackage{fontspec}
\usepackage{xeCJK}
\setmainfont{Times New Roman}
\setCJKmainfont{TaipeiSansTCBeta-Regular}
\XeTeXlinebreaklocale "zh"
\XeTeXlinebreakskip = 0pt plus 1pt

\title{Homework 2 of Introduction to Analysis(II)}
\author{AM15 黃琦翔 111652028}

\begin{document}
\maketitle
\begin{enumerate}
    \item Suppose $f_k(x) = \displaystyle\sum_{n=1}^{k} \dfrac{x}{n^\alpha (1+nx^2)}$.
    First, we observe at the funciton $f_k$, a single term of $f_k$ we call $f_{k, n} = \dfrac{x}{n^{\alpha}(1+nx^2)}$.

    $f_{k, n}'(x) = \dfrac{n^{\alpha}(1+nx^2) - xn^{\alpha}(2nx)}{n^{2\alpha}(1+nx^2)^2} = \dfrac{1-nx^2}{n^{\alpha}(1+nx^2)^2}$ would equal to $0$ at $x = \dfrac{1}{\sqrt{n}}$.
    Thus, the maximun of $|f_{k, n}| = \dfrac{\frac{1}{\sqrt{n}}}{n^{\alpha}(1 + n \cdot \frac{1}{n})} = \dfrac{1}{n^{\alpha + 1/2}}$.
    By $p$-test and $\alpha> \dfrac{1}{2}$, the $f_k(x)$ converges at $x = \dfrac{1}{\sqrt{n}}$.
    And since $f_{k, n}(\dfrac{1}{\sqrt{n}})$ is greater than any else $f_{k, n}(x)$, 
    we can easily know that $f_k(x)$ converges uniformly on $\mathbb{R}$ by Weierstrass M-test.

    \item Since $f_k \to f$ uniformly and $f_k$ are continuous, $f$ is continuous.
    Then, for any $\epsilon > 0$, we have $\delta > 0 $ s.t. if $|y - y'| < \delta$ then $|f(y) - f(y')| < \dfrac{\epsilon}{2}$ for all $y, y' \in \mathbb{R}$.
    Since $x_k \to x$, there exists $N_1 \in \mathbb{N}$ s.t. $|x_k - x| < \delta$ for all $k > N_1$.
    Also we have $N_2 \in \mathbb{N}$ s.t. $|f_k(x) - f(x)| < \dfrac{\epsilon}{2}$ for all $k > N_2$.

    Then, take $N  = \max\{ N_1, N_2\}$, we can get $|f_k(x_k) - f(x)| \leq |f_k(x_k) - f(x_k)| + |f(x_k) - f(x)| = \dfrac{\epsilon}{2} + \dfrac{\epsilon}{2} = \epsilon$ for all $k > N$.
    Thus, $\displaystyle\lim_{k\to\infty} f_k(x_k) = f(x)$.

    \item Since $f_n$ are continuous and converges uniformly, $f$ is continuous(also integrable on $[0, 1]$).
    And since $f$ is continuous, we can find the maximun of $|f|$ on $[0, 1]$ which is called as $M$.
    For any $\epsilon > 0$, there exists $N_1 \in \mathbb{N}$ s.t. $|f_k(x) - f(x)| < \dfrac{\epsilon}{4}$ for all $k > N_1$.
    And there exists $N_2 \in \mathbb{N}$ s.t. $N_2 > \dfrac{M}{2\epsilon}$.
    Then, take $n > N = \max\{N_1, N_2\}$ \begin{align*}
        \left|\int_0^{1-\frac{1}{n}} f_n(x) dx - \int_0^1 f(x) dx\right| &\leq \left|\int_{0}^{1-\frac{1}{n}} f_n(x) - f(x)\ dx\right| + \left|\int_{1-\frac{1}{n}}^{1} f(x)\ dx\right|\\
        &\leq \int_0^1 |f_k(x) - f(x)| dx + \dfrac{1}{n} \cdot M\\
        &\leq 1 \cdot 2\dfrac{\epsilon}{4} + \dfrac{\epsilon}{2}\\
        &= \epsilon
    \end{align*}

    Thus, $\displaystyle\lim_{n\to\infty} \displaystyle\int_{0}^{1-\frac{1}{n}} f_n(x) dx = \displaystyle\int_0^1 f(x) dx$.

    \item Since $|f_k| \leq g$, we have $|f_k(x) - f(x)| \leq 2g(x)$.
    For any $\epsilon > 0$, there exists larger enough $n$ s.t. 
    $\displaystyle\int_{0}^{\frac{1}{n}} 2g(x)\ dx$ and $\displaystyle\int_{n}^{\infty} 2g(x)\ dx$ less than $\dfrac{\epsilon}{3}$.
    And there exists $N \in \mathbb{N}$ s.t. $|f_k(x) - f(x)| < \dfrac{\epsilon}{3(n+1)}$ for all $k > N$.

    Then, \begin{align*}
        \left|\int_{0}^{\infty} f_k(x)\ dx - \int_{0}^{\infty} f(x)\ dx\right| &\leq \int_{0}^{\infty} |f_k(x)- f(x)| dx\\
        &= \int_{0}^{\frac{1}{n}} |f_k(x) - f(x)|\ dx + \int_{\frac{1}{n}}^{n} |f_k(x) - f(x)|\ dx + \int_{n}^{\infty} |f_k(x) - f(x)|\ dx\\
        &\leq \int_{0}^{\frac{1}{n}} 2g(x)\ dx + \int_{n}^{\infty} 2g(x)\ dx + \int_{\frac{1}{n}}^{n} |f_k(x) - f(x)|\ dx\\
        &\leq \dfrac{\epsilon}{3} + \dfrac{\epsilon}{3} + \dfrac{\epsilon}{3}\\
        &= \epsilon
    \end{align*}

    Thus, $\displaystyle\lim_{n\to \infty} \displaystyle\int_{0}^{\infty} f_n(x)\ dx = \displaystyle\int_{0}^{\infty} f(x)\ dx$.
\end{enumerate}
\end{document}