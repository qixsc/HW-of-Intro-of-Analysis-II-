\documentclass[12pt]{article}

\topmargin -40pt
\marginparwidth 0pt
 \oddsidemargin  -40pt
 \evensidemargin 0pt
 \marginparsep 0pt
\textwidth 7.2 in
 \textheight  10 in
 \hoffset  0.1in
\linespread{1.9}

\usepackage{amsthm,amsmath,amssymb,amscd,verbatim,epsfig}
\usepackage{mathptmx}
\usepackage{amsfonts}
%\usepackage{setapace}
\usepackage{graphicx}
\usepackage{bm}
%\usepackage{CJK}
\usepackage{ulem}
\usepackage{multicol}
\usepackage{enumerate}
\usepackage{float}
\usepackage{fontspec}
\usepackage{xeCJK}
\setmainfont{Times New Roman}
\setCJKmainfont{TaipeiSansTCBeta-Regular}
\XeTeXlinebreaklocale "zh"
\XeTeXlinebreakskip = 0pt plus 1pt

\title{Homework 3 of Introduction to Analysis(II)}
\author{AM15 黃琦翔 111652028}

\begin{document}
\maketitle
\begin{enumerate}
    \item For all $\epsilon > 0$, we can find a $N \in \mathbb{N}$ s.t. $|a_n - a| < \dfrac{\epsilon}{2}$ for all $n > N$.
    And we can find a $N' > N$ s.t. $\dfrac{\displaystyle\sum_{i=1}^{N} a_i - a}{N'} < \dfrac{\epsilon}{2}$.
    Thus, for any $n > N'$, \begin{align*}
        |\dfrac{\sum_{i=1}^{n} a_i }{n} - a| &\leq |\dfrac{\sum_{i=1}^{N} a_i - a}{n}| + |\dfrac{\sum_{i=N+1}^{n} a_i - a}{n - N'}|\\
        &< \dfrac{\epsilon}{2} + \dfrac{\epsilon}{2}\\
        &= \epsilon
    \end{align*}
    Therefore, $\displaystyle\lim_{n\to\infty} b_n = a$.

    \item Since $\displaystyle\lim_{n\to\infty} na_n = 0$, by 1., $\displaystyle\lim_{N\to\infty} \displaystyle\sum_{n=0}^{N} a_n = \displaystyle\lim_{N\to\infty} \displaystyle\sum_{n=0}^{N} \dfrac{na_n}{N} = 0$
    Also, since $\displaystyle\lim_{n\to\infty} na_n = 0$, $\displaystyle\lim_{n\to\infty} a_n = 0$.
    Thus, for any $\epsilon > 0$,  there exists a $N \in \mathbb{N}$ s.t. $|\displaystyle\sum_{n=k+1}^{\infty} a_n| < \dfrac{\epsilon}{3}$ for all $k > N$.
    Then, for $x \to 1^{-}$ s.t. $|f(x) - A| < \dfrac{\epsilon}{3}$ and $|\displaystyle\sum_{n=1}^{k} a_n(1-x)| < \dfrac{\epsilon}{3}$, \begin{align*}
        |\sum_{n=0}^{N} a_n - A| &= |\sum_{n=0}^{N} a_n(1-x^n) - \sum_{n=N+1}^{\infty} a_n x^n + (f(x) - A)|\\
        &\leq |\sum_{n=0}^{N} a_n(1-x)| + |\sum_{n=N+1}^{\infty} a_n x^n| + |f(x) - A|\\
        &\leq \dfrac{\epsilon}{3} + \dfrac{\epsilon}{3} + \dfrac{\epsilon}{3} = \epsilon
    \end{align*}
    Therefore, $\displaystyle\sum_{n=0}^{\infty} a_n = A$.

    \newpage
    \item Since $\displaystyle\lim_{x\to 1^-} \displaystyle\sum_{n=1}^{\infty} a_nx^n = A$,
    for any $\epsilon > 0$, we can find a $x_1 \in (0, 1)$ s.t. $|\displaystyle\sum_{n=1}^{\infty} a_nx^n - A| < \dfrac{\epsilon}{3}$ for all $x_1 < x < 1$.
    Then, there exists $N\in\mathbb{N}$ s.t. $|\displaystyle\sum_{n=N'+1}^{\infty} a_n x^n| < \dfrac{\epsilon}{3}$ for all $x$ and $N' > N$.
    And we have $x_2$ s.t. $|\displaystyle\sum_{n=1}^{N} a_n(1-x^n)| < \dfrac{\epsilon}{3}$ for all $x_2 < x < 1$.

    Thus, for any $N' > N$ and $x > \max\{ x_1, x_2\}$,
    \begin{align*}
        |\sum_{n=0}^{N} a_n - A| &\leq |\sum_{n=0}^{N'} a_n(1-x^n)| + |\sum_{n=N'+1}^{\infty} a_n x^n | + |\sum_{n=1}^{\infty} a_nx^n - A|\\
        &< \dfrac{\epsilon}{3} + \dfrac{\epsilon}{3} + \dfrac{\epsilon}{3}\\
        &= \epsilon
    \end{align*}
    Thus, $\displaystyle\sum_{n=1}^{\infty} a_n = A$.

    \item $\ $\begin{enumerate}
        \item[($\implies$)] Since $\displaystyle\sum_{n=1}^{\infty} a_n \sin(nx)$ converges uniformly, 
        there exists a $N \in \mathbb{N}$ s.t. $|\displaystyle\sum_{n=N}^{2N-1} a_n \sin(nx)| < \epsilon$ for all $x$.
        Let $x = \dfrac{1}{2N}$, then $\sin(nx) = \sin(\dfrac{n}{2N})\in [\sin(\frac{1}{2}), 1)$.
        Thus, $\displaystyle\sum_{n=N}^{2N-1} a_n \sin(nx) > \displaystyle\sum_{n=N}^{2N-1} a_n \sin(\dfrac{1}{2}) \geq \displaystyle\sum_{n=N}^{2N-1} a_{2n} \sin(\dfrac{1}{2}) = \dfrac{2n}{2} \sin(\dfrac{1}{2})a_{2n}$.
        Therefore, $2n a_{2n} < \dfrac{2}{\sin(\frac{1}{2})}\epsilon$.
        Using the similar way, we can get $(2n-1) a_{2n-1}\to 0$ as $n\to\infty$ also.
        Hence, we get $na_n\to 0$ as $n\to\infty$.

        \item[($\impliedby$)] Since $na_n \to 0$ as $n\to \infty$, we can find an $N\in\mathbb{N}$ s.t.
        $na_n < \dfrac{\epsilon}{\pi}$ for all $n > N$.
        And since $|\sin(nx)|$ is periodic function, we only need to check the series converges on $[0, \pi]$.
        Then, we want to proof $|\displaystyle\sum_{k=n}^{n+p} a_k \sin(kx)|$ uniformly converges for $n > N$ and $p\in N$.
        We seperate the interval into $[0, \dfrac{\pi}{n+p}],\ [\dfrac{\pi}{n+p}, \dfrac{\pi}{n}],\ [\dfrac{\pi}{n}, \pi]$ three interval.
        
        First one, \begin{align*}
            |\sum_{k=n}^{n+p} a_k \sin(kx)| &\leq \sum_{k=n}^{n+p} a_k \cdot k \cdot x\\
            &\leq \dfrac{k=n}{n+p} k\cdot k \cdot \dfrac{\pi}{n+p}\\
            &\leq 
        \end{align*}
        
    \end{enumerate}
\end{enumerate}
\end{document}