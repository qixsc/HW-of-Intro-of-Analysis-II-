\documentclass[12pt]{article}

\topmargin -40pt
\marginparwidth 0pt
 \oddsidemargin  -40pt
 \evensidemargin 0pt
 \marginparsep 0pt
\textwidth 7.2 in
 \textheight  10 in
 \hoffset  0.1in
\linespread{1.9}

\usepackage{amsthm,amsmath,amssymb,amscd,verbatim,epsfig}
\usepackage{mathptmx}
\usepackage{amsfonts}
%\usepackage{setapace}
\usepackage{graphicx}
\usepackage{bm}
%\usepackage{CJK}
\usepackage{ulem}
\usepackage{multicol}
\usepackage{enumerate}
\usepackage{float}
\usepackage{fontspec}
\usepackage{xeCJK}
\setmainfont{Times New Roman}
\setCJKmainfont{TaipeiSansTCBeta-Regular}
\XeTeXlinebreaklocale "zh"
\XeTeXlinebreakskip = 0pt plus 1pt

\title{Homework 3 of Introduction to Analysis(II)}
\author{AM15 黃琦翔 111652028}

\begin{document}
\maketitle
\begin{enumerate}
    \item For all $\epsilon > 0$, we can find a $N \in \mathbb{N}$ s.t. $|a_n - a| < \dfrac{\epsilon}{2}$ for all $n > N$.
    And we can find a $N' > N$ s.t. $\dfrac{\displaystyle\sum_{i=1}^{N} a_i - a}{N'} < \dfrac{\epsilon}{2}$.
    Thus, for any $n > N'$, \begin{align*}
        |\dfrac{\sum_{i=1}^{n} a_i }{n} - a| &\leq |\dfrac{\sum_{i=1}^{N} a_i - a}{n}| + |\dfrac{\sum_{i=N+1}^{n} a_i - a}{n - N'}|\\
        &< \dfrac{\epsilon}{2} + \dfrac{\epsilon}{2}\\
        &= \epsilon
    \end{align*}
    Therefore, $\displaystyle\lim_{n\to\infty} b_n = a$.

    \item Since $\displaystyle\lim_{n\to\infty} na_n = 0$, by 1., $\displaystyle\lim_{N\to\infty} \displaystyle\sum_{n=0}^{N} a_n = \displaystyle\lim_{N\to\infty} \displaystyle\sum_{n=0}^{N} \dfrac{na_n}{N} = 0$
    Also, since $\displaystyle\lim_{n\to\infty} na_n = 0$, $\displaystyle\lim_{n\to\infty} a_n = 0$.
    Thus, for any $\epsilon > 0$,  there exists a $N \in \mathbb{N}$ s.t. $|\displaystyle\sum_{n=k+1}^{\infty} a_n| < \dfrac{\epsilon}{3}$ for all $k > N$.
    Then, for $x \to 1^{-}$ s.t. $|f(x) - A| < \dfrac{\epsilon}{3}$ and $|\displaystyle\sum_{n=1}^{k} a_n(1-x)| < \dfrac{\epsilon}{3}$, \begin{align*}
        |\sum_{n=0}^{N} a_n - A| &= |\sum_{n=0}^{N} a_n(1-x^n) - \sum_{n=N+1}^{\infty} a_n x^n + (f(x) - A)|\\
        &\leq |\sum_{n=0}^{N} a_n(1-x)| + |\sum_{n=N+1}^{\infty} a_n x^n| + |f(x) - A|\\
        &\leq \dfrac{\epsilon}{3} + \dfrac{\epsilon}{3} + \dfrac{\epsilon}{3} = \epsilon
    \end{align*}
    Therefore, $\displaystyle\sum_{n=0}^{\infty} a_n = A$.

    \newpage
    \item Since $\displaystyle\lim_{x\to 1^-} \displaystyle\sum_{n=1}^{\infty} a_nx^n = A$,
    for any $\epsilon > 0$, we can find a $x_1 \in (0, 1)$ s.t. $|\displaystyle\sum_{n=1}^{\infty} a_nx^n - A| < \dfrac{\epsilon}{3}$ for all $x_1 < x < 1$.
    Then, there exists $N\in\mathbb{N}$ s.t. $|\displaystyle\sum_{n=N'+1}^{\infty} a_n x^n| < \dfrac{\epsilon}{3}$ for all $x$ and $N' > N$.
    And we have $x_2$ s.t. $|\displaystyle\sum_{n=1}^{N} a_n(1-x^n)| < \dfrac{\epsilon}{3}$ for all $x_2 < x < 1$.

    Thus, for any $N' > N$ and $x > \max\{ x_1, x_2\}$,
    \begin{align*}
        |\sum_{n=0}^{N} a_n - A| &\leq |\sum_{n=0}^{N'} a_n(1-x^n)| + |\sum_{n=N'+1}^{\infty} a_n x^n | + |\sum_{n=1}^{\infty} a_nx^n - A|\\
        &< \dfrac{\epsilon}{3} + \dfrac{\epsilon}{3} + \dfrac{\epsilon}{3}\\
        &= \epsilon
    \end{align*}
    Thus, $\displaystyle\sum_{n=1}^{\infty} a_n = A$.

    \item $\ $\begin{enumerate}
        \item[($\implies$)] Since $\displaystyle\sum_{n=1}^{\infty} a_n \sin(nx)$ converges uniformly, 
        there exists a $N \in \mathbb{N}$ s.t. $|\displaystyle\sum_{n=N+1}^{\infty} a_n \sin(nx)| < \epsilon$ for all $x$.
        And assume we can find $x$ s.t. $\sin(nx) = 1$ for all $n>N$, 
        we can rewrite it as 
        $\displaystyle\lim_{N\to\infty}|\displaystyle\sum_{n=N+1}^{\infty} a_n| = \displaystyle\lim_{N\to\infty} \displaystyle\sum_{n=N+1}^{\infty} \dfrac{na_n}{n} = 0$.
        By 1., $\displaystyle\lim_{n\to\infty} na_n = 0$.

        \item[($\impliedby$)] First, we want to proof $\displaystyle\sum_{n=1}^{\infty} \dfrac{\sin(nx)}{n}$ converges uniformly.
        
        Then, since $na_n \to 0$ and $\displaystyle\sum_{n=1}^{\infty} \dfrac{sin(nx)}{n}$ is converges uniformly, 
        by Abel's test again, $\displaystyle\sum_{n=1}^{\infty} na_n \dfrac{\sin(nx)}{n} = \displaystyle\sum_{n=1}^{\infty} a_n\sin(nx)$ converges.
        
        
    \end{enumerate}
\end{enumerate}
\end{document}