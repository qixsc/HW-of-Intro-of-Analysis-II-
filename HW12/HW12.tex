\documentclass[12pt]{article}

\topmargin -40pt
\marginparwidth 0pt
 \oddsidemargin  -40pt
 \evensidemargin 0pt
 \marginparsep 0pt
\textwidth 7.2 in
 \textheight  10 in
 \hoffset  0.1in
\linespread{1.9}

\usepackage{amsthm,amsmath,amssymb,amscd,verbatim,epsfig}
\usepackage{mathptmx}
\usepackage{amsfonts}
%\usepackage{setapace}
\usepackage{graphicx}
\usepackage{bm}
%\usepackage{CJK}
\usepackage{ulem}
\usepackage{multicol}
\usepackage{enumerate}
\usepackage{float}
\usepackage{fontspec}
\usepackage{xeCJK}
\setmainfont{Times New Roman}
\setCJKmainfont{TaipeiSansTCBeta-Regular}
\XeTeXlinebreaklocale "zh"
\XeTeXlinebreakskip = 0pt plus 1pt

\DeclareMathOperator{\volumn}{Vol}
\DeclareMathOperator{\interior}{int}
\DeclareMathOperator{\closure}{cl}

\title{Homework 12 of Introduction to Analysis(II)}
\author{AM15 黃琦翔 111652028}

\begin{document}
\maketitle
\begin{enumerate}
    \item \begin{enumerate}
        \item Since $E$ is Jordan region, $\volumn(\partial E) = 0$.

        \item Since $\closure(E) = \interior(E) \cup \partial E$ and $\interior(E)\subseteq E \subseteq \closure(E)$,

        $\volumn(\closure(E)) = \volumn(\interior(E)) + \volumn(\partial E) = \volumn(\interior(E))\leq \volumn(E) \leq \volumn(\closure(E))$.

        Therefore, $\volumn(\closure(E)) = \volumn(\interior(E)) = E$.

        \item $\ $
        \begin{enumerate}
            \item[$(\implies)$] From $(b)$, we know $\volumn(\interior(E)) = \volumn(E) > 0$, 
            then we can find a set of rectangles $R_n$ s.t. $\sum |R_n| > 0$ and $\cup R_n \subseteq \interior(E)$.
            Therefore, $\interior(E) \neq \emptyset$.
        \end{enumerate}
    \end{enumerate}
\end{enumerate}
\end{document} 