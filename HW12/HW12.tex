\documentclass[12pt]{article}

\topmargin -40pt
\marginparwidth 0pt
 \oddsidemargin  -40pt
 \evensidemargin 0pt
 \marginparsep 0pt
\textwidth 7.2 in
 \textheight  10 in
 \hoffset  0.1in
\linespread{1.9}

\usepackage{amsthm,amsmath,amssymb,amscd,verbatim,epsfig}
\usepackage{mathptmx}
\usepackage{amsfonts}
%\usepackage{setapace}
\usepackage{graphicx}
\usepackage{bm}
%\usepackage{CJK}
\usepackage{ulem}
\usepackage{multicol}
\usepackage{enumerate}
\usepackage{float}
\usepackage{fontspec}
\usepackage{xeCJK}
\setmainfont{Times New Roman}
\setCJKmainfont{TaipeiSansTCBeta-Regular}
\XeTeXlinebreaklocale "zh"
\XeTeXlinebreakskip = 0pt plus 1pt

\DeclareMathOperator{\volumn}{Vol}
\DeclareMathOperator{\interior}{int}
\DeclareMathOperator{\closure}{cl}

\title{Homework 12 of Introduction to Analysis(II)}
\author{AM15 黃琦翔 111652028}

\begin{document}
\maketitle
\begin{enumerate}
    \item \begin{enumerate}
        \item Since $E$ is Jordan region, $\volumn(\partial E) = 0$.

        \item Since $\closure(E) = \interior(E) \cup \partial E$ and $\interior(E)\subseteq E \subseteq \closure(E)$,

        $\volumn(\closure(E)) = \volumn(\interior(E)) + \volumn(\partial E) = \volumn(\interior(E))\leq \volumn(E) \leq \volumn(\closure(E))$.

        Therefore, $\volumn(\closure(E)) = \volumn(\interior(E)) = E$.

        \item $\ $
        \begin{enumerate}
            \item[$(\implies)$] From $(b)$, we know $\volumn(\interior(E)) = \volumn(E) > 0$, 
            then we can find a set of rectangles $R_n$ s.t. $\sum |R_n| > 0$ and $\cup R_n \subseteq \interior(E)$.
            Therefore, $\interior(E) \neq \emptyset$.

            \item[$(\impliedby)$] Since $\interior(E)$ is non-empty, for any $x_0\in \interior(E)$, there exists $\epsilon > 0$ s.t. $D(x_0, \epsilon) \subseteq \interior(E)$.
            Then, we can find a small rectangle $R$ with each length is $\dfrac{\epsilon}{2}$ and $R$ is contained in $D(x_0, \epsilon)$.
            Thus, $\volumn(\interior(E)) > \left(\dfrac{\epsilon}{2}\right)^2 > 0$.
        \end{enumerate}

        \item Since $f$ is continuous, for any $x_0 \in [a, b]$, we can find a sequence $x_k \to x_0$ s.t. $f(x_k) \to f(x_0)$.
        That is, $A = \{(x, f(x)) \mid x\in [a, b]\}$ is closed.
        And since $\partial A \subseteq A$, $\volumn(\partial A) \leq \volumn(A)$.
        
        Since $f$ is continuous on $[a, b]$ is compact, for all $\epsilon > 0$, we can find $\delta > 0$ s.t. $|x-y| < \delta$ implies That $|f(x) - f(y)| < \dfrac{\epsilon}{b-a}$.
        Then, we can find a finite increasing sequence $\{x_i \mid x_i \in [a, b]\}_{i=1}^N$ s.t. $[a, b] \subseteq D(x_i, \dfrac{\delta}{2})$.
        Therefore, for any $y = f(x)$, $y \in D(f(x_i), \dfrac{\epsilon}{b-a})$ for some $i$.

        Then, take $u_0 = a$, $u_i \in D(x_i, \dfrac{\delta}{2})\cap D(x_{i+1}, \dfrac{\delta}{2})$, $u_N = b$,
        and we can get $[a, b] = \cup [u_i, u_{i+1}]$.
        Thus, $A \subseteq \displaystyle\bigcup_{i=0}^N [u_i, u_{i+1}] \times D(\xi_i, \dfrac{\epsilon}{b-a})$ for some $\xi_i \in \{ f(x)\mid x\in [u_i, u_i=1]\}$
        with the sum of the rectangles is $\dfrac{\epsilon}{b-a} \cdot (b-a) = \epsilon$.
        Hence, $\volumn(A) = 0$ and $\volumn(\partial A) = 0$.

        \item 
    \end{enumerate}
\end{enumerate}
\end{document} 