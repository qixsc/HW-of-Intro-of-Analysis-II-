\documentclass[12pt]{article}

\topmargin -40pt
\marginparwidth 0pt
 \oddsidemargin  -40pt
 \evensidemargin 0pt
 \marginparsep 0pt
\textwidth 7.2 in
 \textheight  10 in
 \hoffset  0.1in
\linespread{1.9}

\usepackage{amsthm,amsmath,amssymb,amscd,verbatim,epsfig}
\usepackage{mathptmx}
\usepackage{amsfonts}
%\usepackage{setapace}
\usepackage{graphicx}
\usepackage{bm}
%\usepackage{CJK}
\usepackage{ulem}
\usepackage{multicol}
\usepackage{enumerate}
\usepackage{float}
\usepackage{fontspec}
\usepackage{xeCJK}
\setmainfont{Times New Roman}
\setCJKmainfont{TaipeiSansTCBeta-Regular}
\XeTeXlinebreaklocale "zh"
\XeTeXlinebreakskip = 0pt plus 1pt

\title{Homework 5 of Introduction to Analysis(II)}
\author{AM15 黃琦翔 111652028}

\begin{document}
\maketitle
\begin{enumerate}
    \item \begin{enumerate}
        \item If $x\in A$, $d(x, A) = \| x-x \| = 0$ and $d(x, B) = k > 0$, $\phi(x) = \dfrac{0}{0 + k} = 0$.
        If $x\in B$, $d(x, A) = l > 0$ and $d(x, B) = 0$, $\phi(x) = \dfrac{l}{l + 0} = 1$.
        If $x \notin A \wedge x \notin B$, $d(x, A) = l$ and $d(x, B) = k$, $\phi(x) = \dfrac{l}{l+k} < 1$ and is positive.
        Thus, $0 \leq \phi(x) \leq 1$ for all $x \in \mathbb{R}^n$.

        \item Let $\phi(x) = (b-a)\dfrac{d(x, A)}{d(x, A) + d(x, B)} + a$.
        From (a), we can get $\phi(x\in A) = (b-a)\cdot 0 + a = a$, $\phi(x\in B) = (b-a)\cdot 1 + a = b$,
        and $a \leq phi(x)\leq b$ for all $x\in A$.
    \end{enumerate}

    \item We let $x_n = f^n(x)$, $y_n = f^n(y)$.
    Since $d(x_n, y_n) = d(f(x_{n-1}), f(y_{n-1})) \leq a_1 d(x_{n_1}, y_{n-1})$ for all $x, y$ and $x_n, y_n$ in $S$.
    And $a_n$ converges to $0$, we can get an $a < 1$ s.t. $a_n \leq a^n$.
    Thus, by Contraction Mapping Principle, since $\sup\{ \dfrac{d(f(x), f(y))}{d(x, y)}\} \leq a < 1$, 
    $f$ has unique fixed point.

    \item 
\end{enumerate}
\end{document} 