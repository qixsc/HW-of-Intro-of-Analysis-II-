\documentclass[12pt]{article}

\topmargin -40pt
\marginparwidth 0pt
 \oddsidemargin  -40pt
 \evensidemargin 0pt
 \marginparsep 0pt
\textwidth 7.2 in
 \textheight  10 in
 \hoffset  0.1in
\linespread{1.9}

\usepackage{amsthm,amsmath,amssymb,amscd,verbatim,epsfig}
\usepackage{mathptmx}
\usepackage{amsfonts}
%\usepackage{setapace}
\usepackage{graphicx}
\usepackage{bm}
%\usepackage{CJK}
\usepackage{ulem}
\usepackage{multicol}
\usepackage{enumerate}
\usepackage{float}
\usepackage{fontspec}
\usepackage{xeCJK}
\setmainfont{Times New Roman}
\setCJKmainfont{TaipeiSansTCBeta-Regular}
\XeTeXlinebreaklocale "zh"
\XeTeXlinebreakskip = 0pt plus 1pt

\title{Homework 5 of Introduction to Analysis(II)}
\author{AM15 黃琦翔 111652028}

\begin{document}
\maketitle
\begin{enumerate}
    \item \begin{enumerate}
        \item If $x\in A$, $d(x, A) = \| x-x \| = 0$ and $d(x, B) = k > 0$, $\phi(x) = \dfrac{0}{0 + k} = 0$.
        If $x\in B$, $d(x, A) = l > 0$ and $d(x, B) = 0$, $\phi(x) = \dfrac{l}{l + 0} = 1$.
        If $x \notin A \wedge x \notin B$, $d(x, A) = l$ and $d(x, B) = k$, $\phi(x) = \dfrac{l}{l+k} < 1$ and is positive.

        And for $x, y \in A$ and $\| x- y \| < \delta$, $\phi(x)$
        Thus, $0 \leq \phi(x) \leq 1$ for all $x \in \mathbb{R}^n$.

        \item Let $\phi(x) = (b-a)\dfrac{d(x, A)}{d(x, A) + d(x, B)} + a$.
        From (a), we can get $\phi(x\in A) = (b-a)\cdot 0 + a = a$, $\phi(x\in B) = (b-a)\cdot 1 + a = b$,
        and $a \leq phi(x)\leq b$ for all $x\in A$.
    \end{enumerate}

    \item If $f$ has more than one fixed point, there exists $x, y\in S$ s.t. $d(f^n(x), f^n(y)) = d(x, y)$ for all $n\in\mathbb{N}$(contradiction to $a_n \to 0$).
    Then, we want to show that $f$ has fixed point.
    For any $x_0\in S$, we let $x_k = f^k(x)$ and we can find a $N\in\mathbb{N}$ s.t. $a_n < \epsilon$ for all $n > N$.
    Then, $d(x_{n+k}, x_n) \leq a_n d(x_k, x_0) < \epsilon\cdot d(x_k, x_0)$ for any $k$ and $n > N$.
    Thus, $x_n \to x^*\in S$ by $S$ is complete, and $x^*$ is a fixed point of $f$.

    \item Since $T(u)(t) = \displaystyle\int_a^t u(s)\ ds$, assume $T^m(u)(t) = \displaystyle\int_{a}^{t}\dfrac{(t-s)^{m-1}}{(m-1)!} u(s)\ ds$.
    Then, \begin{align*}
        T^{m+1}(u)(t) &= T(T^m(u))(t)\\
        &= \int_{a}^{t} T^m(u)(s)\ ds\\
        &= \int_{a}^{t} \dfrac{1}{(m-1)!}\int_{a}^{s} (s-\tau)^{m-1} u(\tau)\ d\tau\ ds\\
        &= \int_{a}^{t} \dfrac{1}{(m-1)!}\int_a^\tau (s-\tau)^{m-1} u(\tau)\ ds\ d\tau\\
        &= \int_{a}^{t} \dfrac{1}{m!} (t-\tau)^m u(\tau)\ d\tau
    \end{align*}

    Thus, by M.I., $T^m(u) = \dfrac{1}{(m-1)!}\displaystyle\int_{a}^{t} (t-s)^m u(s) ds \leq \displaystyle\int_{a}^{t} \dfrac{(t-s)^m}{(m-1)!} M\ ds$ 
    for $M$ is upper bound of $u$.

    Thus, $T$ has unique fixed point by 2., and $T(0) = 0$ trivially.

    \item 
    
\end{enumerate}
\end{document} 