\documentclass[12pt]{article}

\topmargin -40pt
\marginparwidth 0pt
 \oddsidemargin  -40pt
 \evensidemargin 0pt
 \marginparsep 0pt
\textwidth 7.2 in
 \textheight  10 in
 \hoffset  0.1in
\linespread{1.9}

\usepackage{amsthm,amsmath,amssymb,amscd,verbatim,epsfig}
\usepackage{mathptmx}
\usepackage{amsfonts}
%\usepackage{setapace}
\usepackage{graphicx}
\usepackage{bm}
%\usepackage{CJK}
\usepackage{ulem}
\usepackage{multicol}
\usepackage{enumerate}
\usepackage{float}
\usepackage{fontspec}
\usepackage{xeCJK}
\setmainfont{Times New Roman}
\setCJKmainfont{TaipeiSansTCBeta-Regular}
\XeTeXlinebreaklocale "zh"
\XeTeXlinebreakskip = 0pt plus 1pt

\title{Homework 5 of Introduction to Analysis(II)}
\author{AM15 黃琦翔 111652028}

\begin{document}
\maketitle
\begin{enumerate}
    \item \begin{enumerate}
        \item If $x\in A$, $d(x, A) = \| x-x \| = 0$ and $d(x, B) = k > 0$, $\phi(x) = \dfrac{0}{0 + k} = 0$.
        If $x\in B$, $d(x, A) = l > 0$ and $d(x, B) = 0$, $\phi(x) = \dfrac{l}{l + 0} = 1$.
        If $x \notin A \wedge x \notin B$, $d(x, A) = l$ and $d(x, B) = k$, $\phi(x) = \dfrac{l}{l+k} < 1$ and is positive.

        And for $x, y \in A$ and $\epsilon > 0$,
        if $\| x- y \| < \delta = \epsilon$, 
        $|d(x, A) - d(x, Y)| < \delta = \epsilon$.
        Thus, $d(x, A)$ is continuous.
        Using the same way, $d(x, B)$ is also continuous.
        Since $\phi(x) = \dfrac{d(x, A)}{d(x, A)+d(x, B)}$ is a continuous function divided by a continuous function which is greater than $0$,
        $\phi(x)$ is also continuous.

        \item Let $\phi(x) = (b-a)\dfrac{d(x, A)}{d(x, A) + d(x, B)} + a$.
        From (a), we can get continuous function $\phi(x\in A) = (b-a)\cdot 0 + a = a$, $\phi(x\in B) = (b-a)\cdot 1 + a = b$,
        and $a \leq \phi(x)\leq b$ for all $x\in A$.
    \end{enumerate}

    \item If $f$ has more than one fixed point, there exists $x, y\in S$ s.t. $d(f^n(x), f^n(y)) = d(x, y)$ for all $n\in\mathbb{N}$(contradiction to $a_n \to 0$).
    Then, we want to show that $f$ has fixed point.
    Since $a_n\to 0$, we can find an $N\in \mathbb{N}$ s.t. $d(f^N(x), f^N(y)) \leq a d(x, y)$ for $a<1$.
    Thus, $f^N$ is a contraction mapping.
    Take $x^*$ be fixed point of $f^N$, that is, $f^N(x^*) = x^*$.
    Then, since $f^{N+1}(x^*) = f(f^N(x^*)) = f(x^*)$ and $f^{N+1} = f^N(f(x^*))$,
    $f(x^*) = f^N(f(x^*))$ is fixed point of $f^N$.
    Since $f^N$ has unique fixed point, $f(x^*) = x^*$.
    Thus, $f$ has unique fixed point.

    \item Since $T(u)(t) = \displaystyle\int_a^t u(s)\ ds$, 
    $\left( T^m(u)(t)\right)' = \left( T(T^{m-1}(u))(t)\right)' = \left( \displaystyle\int_{a}^{t} T^{m-1}(u)(s) ds\right)' = T^{m-1}(u)(t)$ by the FTC.
    And since $T^m(u)(a) = 0$ for all $m\in \mathbb{N}$,
    we can get \begin{align*}
        T^{m}(u)(t) &= T(T^{m-1}(u))(t)\\
        &= \int_{a}^{t} T^{m-1}(u)(s)\ ds\\
        &= -\int_{a}^{t} T^{m-1}(u)(s)\ d(t-s)\\
        &= - (t-s) T^{m-1}(u)(s)\mid_a^t + \int_{a}^{t} (t-s)\ d(T^{m-1}(u)(s))\\
        &= \int_{a}^{t} (t-s)T^{m-2}(u)(s)\\
        &= \int_{a}^{t} T^{m-2}(u)(s)\ d(\dfrac{-(t-s)^2}{2})\\
        &= \int_{a}^{t} \dfrac{(t-s)^2}{2!} T^{m-3}(u)(s)\ ds\\
        &\vdots\\
        &= \int_{a}^{t} \dfrac{(t-s)^{m-1}}{(m-1)!} \ d(T(u)(s))\\
        &= \int_{a}^{t} \dfrac{(t-s)^{m-1}}{(m-1)!} u(s)\ ds
    \end{align*}

    Thus, $T^m(u) = \dfrac{1}{(m-1)!}\displaystyle\int_{a}^{t} (t-s)^{m-1} u(s) ds \leq \displaystyle\int_{a}^{t} \dfrac{(t-s)^{m-1}}{(m-1)!} M\ ds \leq \displaystyle\int_{a}^{t} \dfrac{M\cdot (t-a)^{m-1}}{(m-1)!}\ ds$
    with $M$ is upper bound of $u$. Then, taking $a_n = \dfrac{(b-a)^n}{(n-1)!}$, $d(T^n(u), T^n(v)) \leq a_n d(u, v)$ and $a_n \to 0$.
    Therefore, by 2., $T$ has unique fixed point, and $T(0) = 0$ trivially.

    \item Since there exists $P_n$ is polynomial on $[0, 1]$ with degree $n$ and $P_n \to f$,
    \begin{align*}
        \int_{0}^{1}
    \end{align*}
\end{enumerate}
\end{document} 