\documentclass[12pt]{article}

\topmargin -40pt
\marginparwidth 0pt
 \oddsidemargin  -40pt
 \evensidemargin 0pt
 \marginparsep 0pt
\textwidth 7.2 in
 \textheight  10 in
 \hoffset  0.1in
\linespread{1.9}

\usepackage{amsthm,amsmath,amssymb,amscd,verbatim,epsfig}
\usepackage{mathptmx}
\usepackage{amsfonts}
%\usepackage{setapace}
\usepackage{graphicx}
\usepackage{bm}
%\usepackage{CJK}
\usepackage{ulem}
\usepackage{multicol}
\usepackage{enumerate}
\usepackage{float}
\usepackage{fontspec}
\usepackage{xeCJK}
\setmainfont{Times New Roman}
\setCJKmainfont{TaipeiSansTCBeta-Regular}
\XeTeXlinebreaklocale "zh"
\XeTeXlinebreakskip = 0pt plus 1pt

\DeclareMathOperator{\volume}{Vol}
\DeclareMathOperator{\interior}{int}
\DeclareMathOperator{\closure}{cl}
\newcommand{\boundary}{\partial}

\title{Homework 14 of Introduction to Analysis(II)}
\author{AM15 黃琦翔 111652028}

\begin{document}
\maketitle
\begin{enumerate}
    \item Let $A_M = \{x \in E \mid f_M\text{ is discontinuous at } x\}$ and $A = \{ x\in E \mid f\text{ is discontinuous at } x\}$.
    For any $M \in \mathbb{N}$, if $x$ is a point that $f_M$ is discontinuous at $x$, 
    That means $f(x) \leq M$ and $f$ is discontinuous at $x$.
    That is, $A_M \subseteq A$ for all $M$ implies that $\cup A_m \subseteq A$.

    And for any $x \in A$, there exists a $N \in \mathbb{N}$ s.t. $f(x) < N$.
    Then, $x \in A_N$.
    Therefore, $A = \cup A_M$.

    \item Let $\displaystyle\int_0^1 f(x)\ dx = \alpha$ and $\displaystyle\sup_{x\in [0, 1]} \sup f(x) = M$.
    And since $f$ is integrable, there exists a partition $P$ s.t. $\int_{0}^{1} f(x)\ dx - L(f, P) \leq \dfrac{\alpha}{4}$.



    \item Since $\displaystyle\int_{E} f(x)\ dx = 0$, then for any $\epsilon > 0$,
    there exists rectangles $R_n$ that (U)$\displaystyle\int_{E} f(x)\ dx = \sum \displaystyle\sup_{x \in R_i} f(x) \cdot |R_i| < \epsilon$.
    That is, 
\end{enumerate}
\end{document} 