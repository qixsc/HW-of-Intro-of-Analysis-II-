\documentclass[12pt]{article}

\topmargin -40pt
\marginparwidth 0pt
 \oddsidemargin  -40pt
 \evensidemargin 0pt
 \marginparsep 0pt
\textwidth 7.2 in
 \textheight  10 in
 \hoffset  0.1in
\linespread{1.9}

\usepackage{amsthm,amsmath,amssymb,amscd,verbatim,epsfig}
\usepackage{mathptmx}
\usepackage{amsfonts}
%\usepackage{setapace}
\usepackage{graphicx}
\usepackage{bm}
%\usepackage{CJK}
\usepackage{ulem}
\usepackage{multicol}
\usepackage{enumerate}
\usepackage{float}
\usepackage{fontspec}
\usepackage{xeCJK}
\setmainfont{Times New Roman}
\setCJKmainfont{TaipeiSansTCBeta-Regular}
\XeTeXlinebreaklocale "zh"
\XeTeXlinebreakskip = 0pt plus 1pt

\DeclareMathOperator{\volume}{Vol}
\DeclareMathOperator{\interior}{int}
\DeclareMathOperator{\closure}{cl}
\newcommand{\boundary}{\partial}

\title{Homework 14 of Introduction to Analysis(II)}
\author{AM15 黃琦翔 111652028}

\begin{document}
\maketitle
\begin{enumerate}
    \item Let $A_M = \{x \in E \mid f_M\text{ is discontinuous at } x\}$ and $A = \{ x\in E \mid f\text{ is discontinuous at } x\}$.
    For any $M \in \mathbb{N}$, if $x$ is a point that $f_M$ is discontinuous at $x$, 
    That means $f(x) \leq M$ and $f$ is discontinuous at $x$.
    That is, $A_M \subseteq A$ for all $M$ implies that $\cup A_m \subseteq A$.

    And for any $x \in A$, there exists a $N \in \mathbb{N}$ s.t. $f(x) < N$.
    Then, $x \in A_N$.
    Therefore, $A = \cup A_M$.

    \item Let $\displaystyle\int_0^1 f(x)\ dx = \alpha$ and $\displaystyle\sup_{x\in [0, 1]} \sup f(x) = M$.
    
    By $1$. , 
    
    Since $f$ is integrable, there exists a partition $P$ s.t. $\int_{0}^{1} f(x)\ dx - L(f, P) \leq \dfrac{\alpha}{4}$.

    \item Let $A = \{x \in E\mid f(x)\neq 0\}$. If $A$ is empty, then $A$ is measure zero.
    
    Suppose $A$ is non-empty.
    Then, for a large enough $N\in \mathbb{N}$, $A_N = \{x\in E\mid f(x) > \dfrac{1}{N}\}$.
    Using the same argument of $1$. , we can have $A_N \to A$ as $N \to \infty$.
    Since $\displaystyle\int_{E} f(x)\ dx = 0$, $\displaystyle\int_{A_N} f(x)\ dx = 0$.
    Thus, for any $\epsilon > 0$, there exists rectangles such that $\dfrac{1}{N}\sum |R_i| \leq (L)\displaystyle\int_{A_N} f(x)\ dx \leq \dfrac{\epsilon}{N}$.
    Therefore, $\sum |R_i| < N \cdot \dfrac{\epsilon}{N} = \epsilon$ and $A_N$ is measure zero.
    
    By the theorem that countable set of measure zero is also measure zero, we can have $A$ is measure zero.
\end{enumerate}
\end{document} 