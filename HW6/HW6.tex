\documentclass[12pt]{article}

\topmargin -40pt
\marginparwidth 0pt
 \oddsidemargin  -40pt
 \evensidemargin 0pt
 \marginparsep 0pt
\textwidth 7.2 in
 \textheight  10 in
 \hoffset  0.1in
\linespread{1.9}

\usepackage{amsthm,amsmath,amssymb,amscd,verbatim,epsfig}
\usepackage{mathptmx}
\usepackage{amsfonts}
%\usepackage{setapace}
\usepackage{graphicx}
\usepackage{bm}
%\usepackage{CJK}
\usepackage{ulem}
\usepackage{multicol}
\usepackage{enumerate}
\usepackage{float}
\usepackage{fontspec}
\usepackage{xeCJK}
\setmainfont{Times New Roman}
\setCJKmainfont{TaipeiSansTCBeta-Regular}
\XeTeXlinebreaklocale "zh"
\XeTeXlinebreakskip = 0pt plus 1pt

\title{Homework 6 of Introduction to Analysis(II)}
\author{AM15 黃琦翔 111652028}

\begin{document}
\maketitle
\begin{enumerate}
    \item Let $\phi(x) = \arctan(x)$, then $\phi \circ f(x) = \phi(f(x))$ is bdd by $(-\dfrac{\pi}{2}, \dfrac{\pi}{2})$.
    Then, by Tietze's Extension Theorem, there exists $g \in C(\mathbb{R}^n, \mathbb{R})$ s.t. 
    $g(x) = \phi(f(x))$ for all $x\in D$ and $\sup|g(x)| = \sup |\phi(f(x))|\leq \dfrac{\pi}{2}$.

    Thus, there exists $h(x) = \tan(g(x))$ in $C(\mathbb{R}^n, \mathbb{R})$ and $h(x) = f(x)$ for all $x\in D$ by 
    $\phi(x)$ is invertible.

    \item Since $\dfrac{\partial f_i}{\partial x}$ exists for all $x$ and all $i$, 
    we get the Jacobian matrix $$\left[J(x)\right] = \begin{bmatrix}
        \dfrac{\partial f_1}{\partial x}(x)\\
        \dfrac{\partial f_2}{\partial x}(x)\\
        \vdots\\
        \dfrac{\partial f_m}{\partial x}(x)\\
    \end{bmatrix}.$$
    Then, we want to show that $J = Df$.
    Since $h \to 0$, $\dfrac{f_i}{\partial x}(x)*h = f_i(x+h) - f_i(x)$, then for any $x$, 

    $\displaystyle\lim_{h\to 0} \dfrac{\| f(x+h) - f(x) - [J(x)]h\|}{|h|}= \displaystyle\lim_{h\to 0} \dfrac{\| (f_1(x+h) - f(x) - [J(x)]_i h,\ \cdots,\ f_m(x+h) - f_m(x) - [J(x)]_m h)\|}{|h|}\\
    = \dfrac{\| (0, 0, \cdots, 0)\|}{|h|} = 0$.
    Therefore, $Df = J$ exists.

    \item 
\end{enumerate}
\end{document} 