\documentclass[12pt]{article}

\topmargin -40pt
\marginparwidth 0pt
 \oddsidemargin  -40pt
 \evensidemargin 0pt
 \marginparsep 0pt
\textwidth 7.2 in
 \textheight  10 in
 \hoffset  0.1in
\linespread{1.9}

\usepackage{amsthm,amsmath,amssymb,amscd,verbatim,epsfig}
\usepackage{mathptmx}
\usepackage{amsfonts}
%\usepackage{setapace}
\usepackage{graphicx}
\usepackage{bm}
%\usepackage{CJK}
\usepackage{ulem}
\usepackage{multicol}
\usepackage{enumerate}
\usepackage{float}
\usepackage{fontspec}
\usepackage{xeCJK}
\setmainfont{Times New Roman}
\setCJKmainfont{TaipeiSansTCBeta-Regular}
\XeTeXlinebreaklocale "zh"
\XeTeXlinebreakskip = 0pt plus 1pt

\title{Homework 10 of Introduction to Analysis(II)}
\author{AM15 黃琦翔 111652028}

\begin{document}
\maketitle
\begin{enumerate}
    \item \begin{enumerate}
        \item $f'(0) = \displaystyle\lim_{h\to 0} \dfrac{f(h) - f(0)}{h} = \displaystyle\lim_{h\to 0} 1 + 2h\sin(\dfrac{1}{h}) = 1$.
        
        \item $\displaystyle\lim_{x\to 0} f(x) = \displaystyle\lim_{h\to \infty} \dfrac{1}{h} + 2\dfrac{\sin(h)}{h^2} = 0$.
        Thus, $f$ is continuous on $0$.
        And for any $x$ close to $0$, $|f(x+h) - f(x)| \leq |h| + 2|(x+h)^2| + 2|x^2| \to 0$ as $h, x \to 0$, $f$ is continuous on a small interval $I_1$.

        Since $f'(x) = 1 + 2(2x\sin(\dfrac{1}{x}) - \cos(\dfrac{1}{x}))$, there exsits $x$ in any interval contains $0$ s.t. $f'(x)< 0$.
        Therefore, $f$ is neither increasing nor decreasing and not one to one in any interval contains $0$.
        Thus, $f$ is not invertible near $0$.

        \item This is not contradict to inverse function theorem.
    \end{enumerate}

    \item \begin{align*}
        \| f(x_1) - f(x_2) - (x_1 - x_2)\| &= \| x_1 + g(x_1) - (x_2 + g(x_2)) - (x_1 - x_2) \| \\
        &= \| g(x_1) - g(x_2)\|\\
        &\leq a\| x_1 - x_2\|
    \end{align*}

    Then, for any $x_1, x_2\in \mathbb{R}^n$ and $x_1 \neq x_2$, $\| f(x_1) - f(x_2)\| \geq \|f(x_1) - f(x_2) - (x_1 - x_2)\| + \| x_1 - x_2\| \geq (1-a)\|x_1 - x_2\| > 0$.
    Thus, $f$ is one-to-one.

    And we can get $g$ is a contraction mapping from $\mathbb{R}^n$ to $\mathbb{R}^n$,
    we can find an unique fixed point $x^*$ s.t. $g(x^*) = x^*$ and $f(x^*) = 2x^*$.
    Then, for any $y \in \mathbb{R}^n$, 
   
    \item 

\end{enumerate}
\end{document} 