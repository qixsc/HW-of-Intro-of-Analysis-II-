\documentclass[12pt]{article}

\topmargin -40pt
\marginparwidth 0pt
 \oddsidemargin  -40pt
 \evensidemargin 0pt
 \marginparsep 0pt
\textwidth 7.2 in
 \textheight  10 in
 \hoffset  0.1in
\linespread{1.9}

\usepackage{amsthm,amsmath,amssymb,amscd,verbatim,epsfig}
\usepackage{mathptmx}
\usepackage{amsfonts}
%\usepackage{setapace}
\usepackage{graphicx}
\usepackage{bm}
%\usepackage{CJK}
\usepackage{ulem}
\usepackage{multicol}
\usepackage{enumerate}
\usepackage{float}
\usepackage{fontspec}
\usepackage{xeCJK}
\setmainfont{Times New Roman}
\setCJKmainfont{TaipeiSansTCBeta-Regular}
\XeTeXlinebreaklocale "zh"
\XeTeXlinebreakskip = 0pt plus 1pt

\title{Homework 10 of Introduction to Analysis(II)}
\author{AM15 黃琦翔 111652028}

\begin{document}
\maketitle
\begin{enumerate}
    \item \begin{enumerate}
        \item $f'(0) = \displaystyle\lim_{h\to 0} \dfrac{f(h) - f(0)}{h} = \displaystyle\lim_{h\to 0} 1 + 2h\sin(\dfrac{1}{h}) = 1$.
        
        \item $\displaystyle\lim_{x\to 0} f(x) = \displaystyle\lim_{h\to \infty} \dfrac{1}{h} + 2\dfrac{\sin(h)}{h^2} = 0$.
        Thus, $f$ is continuous on $0$.
        And for any $x$ close to $0$, $|f(x+h) - f(x)| \leq |h| + 2|(x+h)^2| + 2|x^2| \to 0$ as $h, x \to 0$, $f$ is continuous on a small interval $I_1$.

        Since $f'(x) = 1 + 2(2x\sin(\dfrac{1}{x}) - \cos(\dfrac{1}{x}))$, there exsits $x$ in any interval contains $0$ s.t. $f'(x)< 0$.
        Therefore, $f$ is neither increasing nor decreasing and not one to one in any interval contains $0$.
        Thus, $f$ is not invertible near $0$.

        \item This is not contradict to inverse function theorem since $f'$ is not continuous.
    \end{enumerate}

    \item \begin{align*}
        \| f(x_1) - f(x_2) - (x_1 - x_2)\| &= \| x_1 + g(x_1) - (x_2 + g(x_2)) - (x_1 - x_2) \| \\
        &= \| g(x_1) - g(x_2)\|\\
        &\leq a\| x_1 - x_2\|
    \end{align*}

    And for any $x\in \mathbb{R}^n$, $\|Df(x)\| = \|I + Dg(x)\| \geq (1-a)$,
    then $\| f(x) - f(y)\| = \|x-y\| - \| f(x) - f(y) - (x-y)\| \geq (1-a)\|x-y\| > 0$ for $x, y\in \mathbb{R}^n$, $x\neq y$ and $c$ on the line between $x, y$.
    Thus, $f(x) \neq f(y)$ implies that $f$ is one to one.

    Let $h_y(x) = y - g(x)$, then $\| h_y(x_1) - h_y(x_2)\| = \| g(x_1) - g(x_2) \| \leq a\|x_1 - x_2\|$.
    Thus, $h_y$ is contraction mapping and exsits unique fixed point $x^*$ s.t. $x^* = h_y(x^*) = y - g(x^*)$.
    \newpage
    Then, for any $y$, exsits $x^*$ s.t. $y = x^* + g(x^*) = f(x^*)$.
    Therefore, $f$ is surjective.
    Hence, $f$ is bijective.
   
    \item Since $\|f(x) - f(y) \| \geq C\| x-y \|$ and $C > 0$, if $x \neq y$, $\| f(x) - f(y) \| \geq C\|x - y\| > 0$.
    Then, $f$ is injective.
    And since $f$ is continuous, $f(\mathbb{R}^n)$ is closed since $\mathbb{R}^n$ is closed.

    Now proof $f(\mathbb{R}^n)$ is open.
    For any $x_0$ and $\epsilon \in (0, C)$, since $f$ is differentiable, there exsits $h$ s.t. 
    $\| f(x_0+h) - f(x_0) - Df(x_0)(h)\| < \epsilon \| h\|$.
    Then, since $\| f(x_0+h) - f(x_0)\| - \|Df(x_0)(h)\| \leq \| f(x_0+h) - f(x_0) - Df(x_0)(h)\|$,
    $C\| h\| - \|Df(x_0)(h)\| < \epsilon\|h\| < C\| h \|$.
    Thus, $\| Df(x_0)(h)\| > 0$.
    Therefore, by inverse function theorem, there exsits a neighborhood $U$ near $x_0$ s.t. $f(U)$ is open.
    Then, $f(\mathbb{R}^n)$ is open.

    Therefore, $f$ is bijective and then $f$ is invertible.
    And by inverse function theorem, $f^{-1}$ is differentiable and continuous on $\mathbb{R}^n$.
    
\end{enumerate}
\end{document} 