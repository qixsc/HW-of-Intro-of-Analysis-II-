\documentclass[12pt]{article}

\topmargin -40pt
\marginparwidth 0pt
 \oddsidemargin  -40pt
 \evensidemargin 0pt
 \marginparsep 0pt
\textwidth 7.2 in
 \textheight  10 in
 \hoffset  0.1in
\linespread{1.9}

\usepackage{amsthm,amsmath,amssymb,amscd,verbatim,epsfig}
\usepackage{mathptmx}
\usepackage{amsfonts}
%\usepackage{setapace}
\usepackage{graphicx}
\usepackage{bm}
%\usepackage{CJK}
\usepackage{ulem}
\usepackage{multicol}
\usepackage{enumerate}
\usepackage{float}
\usepackage{fontspec}
\usepackage{xeCJK}
\setmainfont{Times New Roman}
\setCJKmainfont{TaipeiSansTCBeta-Regular}
\XeTeXlinebreaklocale "zh"
\XeTeXlinebreakskip = 0pt plus 1pt

\title{Homework 9 of Introduction to Analysis(II)}
\author{AM15 黃琦翔 111652028}

\begin{document}
\maketitle
\begin{enumerate}
    \item By observe, $\Delta(x) = \det(D^2f(x))\geq 0$.
    \begin{enumerate}
        \item Since $\Delta(x) \geq 0$ and $f_{xx}(x) > 0$, $D^2f(x)$ is positive semi-definite on $D(c, \delta)$.
        Thus, for any $x \in D(c, \delta)$, \begin{align*}
            Df(x) &= Df(x) - Df(c)\\
            &= \int_{0}^{1} D^2f(c+t(x-c))(x-c)\ dt\\
            &\geq \int_{0}^{1} 0(x-c)\ dt\\
            &\geq 0
        \end{align*}
        
        Then, \begin{align*}
            f(x) - f(c) &= \int_{0}^1 Df(c + t(x-c))(x-c)\ dt\\
            &\geq \int_{0}^{1} 0(x-c)\ dt\\
            &\geq 0
        \end{align*}

        Threrfore, $c$ is a local minimun point of $f$. 

        \item Since $f_{xx}(x) < 0$ and $\Delta(x) \geq 0$, $D^f(x)$ is negative semi-definite.
        Using the same argument as (a), we can get $f(x) - f(c) \leq 0$ for all $x\in \| x-c\| < \delta$.
        Thus, $c$ is local maximun point of $f$.

    \end{enumerate}

    \newpage
    \item Since $f_x(x_0) = f_y(x_0) = 0$ and $f \in C^2(V, \mathbb{R})$, $Df(x_0) = 0$ and $x_0$ is a critical point.
    
    If $f_{xy}(x_0)\neq 0$, $\det(D^2f(x_0)) = - (f_{xy}(x_0))^2 < 0$, then $D^2f(x_0)$ is not positive definite and negative definite.
    Thus, $x_0$ will be saddle point.

    \item Since $D^rf(x)$ is continuous, $M = \sup\{ \| D^rf(x) \| \mid x \in E \text{ and in neighberhood of } x_0\}$.
    Thus, $\displaystyle\lim_{h\to 0} R_{r-1}(x_0, h) = \displaystyle\lim_{h\to 0} \dfrac{D^rf(x_0)(h, h, \cdots, h)}{r!\| h\|^{r-1}} \leq \displaystyle\lim_{h\to 0} \dfrac{M \cdot \| h\|^r}{r!\|h\|^{r-1}} = \displaystyle\lim_{h\to 0} \dfrac{M}{r!}\|h\| =0$.

\end{enumerate}
\end{document} 