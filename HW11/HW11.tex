\documentclass[12pt]{article}

\topmargin -40pt
\marginparwidth 0pt
 \oddsidemargin  -40pt
 \evensidemargin 0pt
 \marginparsep 0pt
\textwidth 7.2 in
 \textheight  10 in
 \hoffset  0.1in
\linespread{1.9}

\usepackage{amsthm,amsmath,amssymb,amscd,verbatim,epsfig}
\usepackage{mathptmx}
\usepackage{amsfonts}
%\usepackage{setapace}
\usepackage{graphicx}
\usepackage{bm}
%\usepackage{CJK}
\usepackage{ulem}
\usepackage{multicol}
\usepackage{enumerate}
\usepackage{float}
\usepackage{fontspec}
\usepackage{xeCJK}
\setmainfont{Times New Roman}
\setCJKmainfont{TaipeiSansTCBeta-Regular}
\XeTeXlinebreaklocale "zh"
\XeTeXlinebreakskip = 0pt plus 1pt

\title{Homework 11 of Introduction to Analysis(II)}
\author{AM15 黃琦翔 111652028}

\begin{document}
\maketitle
\begin{enumerate}
    \item 

    \item Suppose $f \in C^1(\mathbb{R}^2, \mathbb{R})$, $Df(x_0, y_0) \neq 0$ for some $x_0, y_0$
    (or $f$ is constant function and not one-to-one).
    Then, suppose $\dfrac{\partial f}{\partial x} \neq 0$ for neighborhood of $(x_0, y_0)$,
    and let $h(x, y) = f(x, y) - f(x_0, y_0)$ with $\dfrac{\partial h}{\partial x} \neq 0$.
    by Implicit Function Theorem, there is a neighborhood $U \subseteq \mathbb{R}^2$ and $W \subseteq \mathbb{R}$ s.t. $(x_0, y_0) \in U$ and $y_0 \in W$ 
    and a function $g: W \to \mathbb{R}^2$ s.t. $h(g(y), y) = f(g(y), y) - f(x_0, y_0) = 0$.
    Then, $f(g(y), y) = f(x_0, y_0)$ for $y \in W$ and $f$ is not one-to-one.

    If $\dfrac{\partial f}{\partial x} = 0$, then $\dfrac{\partial f}{\partial y} \neq 0$ and use the same argument can get the same result.

    \item 
    
\end{enumerate}
\end{document} 