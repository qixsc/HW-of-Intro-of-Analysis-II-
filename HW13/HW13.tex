\documentclass[12pt]{article}

\topmargin -40pt
\marginparwidth 0pt
 \oddsidemargin  -40pt
 \evensidemargin 0pt
 \marginparsep 0pt
\textwidth 7.2 in
 \textheight  10 in
 \hoffset  0.1in
\linespread{1.9}

\usepackage{amsthm,amsmath,amssymb,amscd,verbatim,epsfig}
\usepackage{mathptmx}
\usepackage{amsfonts}
%\usepackage{setapace}
\usepackage{graphicx}
\usepackage{bm}
%\usepackage{CJK}
\usepackage{ulem}
\usepackage{multicol}
\usepackage{enumerate}
\usepackage{float}
\usepackage{fontspec}
\usepackage{xeCJK}
\setmainfont{Times New Roman}
\setCJKmainfont{TaipeiSansTCBeta-Regular}
\XeTeXlinebreaklocale "zh"
\XeTeXlinebreakskip = 0pt plus 1pt

\DeclareMathOperator{\volume}{Vol}
\DeclareMathOperator{\interior}{int}
\DeclareMathOperator{\closure}{cl}
\newcommand{\boundary}{\partial}

\title{Homework 13 of Introduction to Analysis(II)}
\author{AM15 黃琦翔 111652028}

\begin{document}
\maketitle
\begin{enumerate}
    \item \begin{enumerate}
        \item For any $\epsilon > 0$, 
        we can find an open cube $D(x_0, \delta)$ that $R_j \subseteq D(x_0, \delta)$ and 

        \item Since for any $E_i$ is measure zero, we can find $\{ R_{i, j}\}_j$ s.t. 
        $\displaystyle\sum_{j} |R_{i, j}| < \dfrac{\epsilon}{2^{i}}$.
        Then, let $R_{1, 1} = R_1$, $R_{1,2} = R_2$, $R_{2, 1} = R_3$, $R_{1, 3} = R_4$, $R_{2, 2} = R_5$ etc.
        And we can get $\{ R_k\}_k$ contains $\displaystyle\bigcup_i E_i$ and $\displaystyle\sum_{k} |R_k| < \epsilon(\dfrac{1}{2} + \dfrac{1}{4} + \cdots) = \epsilon$.
        Therefore, union of countable measure zero sets is measure zero.

        \item Since $A$ is compact and (a), we can find finite open cubes which covers $A$ and volume of their union is less than $\epsilon$.
        Thus, take union of closure on each open cube and we can find a retangle covers $A$ with volume less than $\epsilon$.
        Therefore, volume of $A$ is $0$.
    \end{enumerate}

    \item For any grid $g$, $U(1, g) = L(1, g)$. Then, $1$ is integrable.
    And for any $\epsilon > 0$, there exists rectangles $R_i$ s.t. $|\sum |R_i| - \volume(E)| <\epsilon$.
    Thus, $\displaystyle\int_E 1\ dE \leq U(1, g) = \sum 1 \cdot |R_i| < \volume(E) + \epsilon$.
    Also, $\displaystyle\int_E 1\ dE \geq \volume(E) - \epsilon$.
    Therefore, $\displaystyle\int_E 1\ dE = \volume(E)$.

    \item \begin{enumerate}
        \item If $x$ is rational, $G(x, t) = \displaystyle\int_{0}^{t} 1\ dy = t$.
        IF $x$ is irrational, $G(x, t) = \displaystyle\int_{0}^{t} 2y\ dy = t^2$.

        Since $t \in [0, 1]$, $t^2 \leq t$.
        Then, (U)$\displaystyle\int_{0}^{1} G(x, t)\ dx = t$ and (L)$\displaystyle\int_{0}^{1} G(x, t)\ dx = t^2$.

        Therefore, $\displaystyle\int_{0}^{1} (\displaystyle\int_{0}^{1} f(x, y)\ dy)\ dx = \displaystyle\int_{0}^{1} G(x, 1)\ dx = \displaystyle\int_{0}^{1} 1\ dx = 1$.

        \item As $y \to 0$, (U)$\displaystyle\int_{0}^{1} f(x, y)\ dx = 1$ but (L)$\displaystyle\int_{0}^{1} f(x, y)\ dx = 2y \to 0$.
        Thus, $\displaystyle\int_{Q} f(X)\ dx$ doesn't exists.
    \end{enumerate}
\end{enumerate}
\end{document} 