\documentclass[12pt]{article}

\topmargin -40pt
\marginparwidth 0pt
 \oddsidemargin  -40pt
 \evensidemargin 0pt
 \marginparsep 0pt
\textwidth 7.2 in
 \textheight  10 in
 \hoffset  0.1in
\linespread{1.9}

\usepackage{amsthm,amsmath,amssymb,amscd,verbatim,epsfig}
\usepackage{mathptmx}
\usepackage{amsfonts}
%\usepackage{setapace}
\usepackage{graphicx}
\usepackage{bm}
%\usepackage{CJK}
\usepackage{ulem}
\usepackage{multicol}
\usepackage{enumerate}
\usepackage{float}
\usepackage{fontspec}
\usepackage{xeCJK}
\setmainfont{Times New Roman}
\setCJKmainfont{TaipeiSansTCBeta-Regular}
\XeTeXlinebreaklocale "zh"
\XeTeXlinebreakskip = 0pt plus 1pt

\DeclareMathOperator{\volume}{Vol}
\DeclareMathOperator{\interior}{int}
\DeclareMathOperator{\closure}{cl}
\newcommand{\boundary}{\partial}

\title{Homework 13 of Introduction to Analysis(II)}
\author{AM15 黃琦翔 111652028}

\begin{document}
\maketitle
\begin{enumerate}
    \item \begin{enumerate}
        \item For any $\epsilon > 0$, 
        we can find an open cube $D(x_0, \delta)$ that $R_j \subseteq D(x_0, \delta)$ and 

        \item Since for any $E_i$ is measure zero, we can find $\{ R_{i, j}\}_j$ s.t. 
        $\displaystyle\sum_{j} |R_{i, j}| < \dfrac{\epsilon}{2^{i}}$.
        Then, let $R_{1, 1} = R_1$, $R_{1,2} = R_2$, $R_{2, 1} = R_3$, $R_{1, 3} = R_4$, $R_{2, 2} = R_5$ etc.
        And we can get $\{ R_k\}_k$ contains $\displaystyle\bigcup_i E_i$ and $\displaystyle\sum_{k} |R_k| < \epsilon(\dfrac{1}{2} + \dfrac{1}{4} + \cdots) = \epsilon$.
        Therefore, union of countable measure zero sets is measure zero.

        \item 

    \end{enumerate}
\end{enumerate}
\end{document} 