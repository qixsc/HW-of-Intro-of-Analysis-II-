\documentclass[12pt]{article}

\topmargin -40pt
\marginparwidth 0pt
 \oddsidemargin  -40pt
 \evensidemargin 0pt
 \marginparsep 0pt
\textwidth 7.2 in
 \textheight  10 in
 \hoffset  0.1in
\linespread{1.9}

\usepackage{amsthm,amsmath,amssymb,amscd,verbatim,epsfig}
\usepackage{mathptmx}
\usepackage{amsfonts}
%\usepackage{setapace}
\usepackage{graphicx}
\usepackage{bm}
%\usepackage{CJK}
\usepackage{ulem}
\usepackage{multicol}
\usepackage{enumerate}
\usepackage{float}
\usepackage{fontspec}
\usepackage{xeCJK}
\setmainfont{Times New Roman}
\setCJKmainfont{TaipeiSansTCBeta-Regular}
\XeTeXlinebreaklocale "zh"
\XeTeXlinebreakskip = 0pt plus 1pt

\title{Homework 8 of Introduction to Analysis(II)}
\author{AM15 黃琦翔 111652028}

\begin{document}
\maketitle
\begin{enumerate}
    \item $f_x(0, 0) = \displaystyle\lim_{x\to 0} \dfrac{f(x, 0) - f(0)}{x} = \displaystyle\lim_{x\to 0} \dfrac{x\cdot 0(x^2-0^2)/(x^2+0^2)}{x} = 0$ exists.
    Also, we can esaily get that $f_y(0, 0) = 0$.
    Then, $f_{xy}(0, 0) = \dfrac{\partial f_x}{\partial y}(0, 0) = \displaystyle\lim_{y\to 0} \dfrac{y(0^4+4\cdot 0^2y^2 - y^4)/(0^2 + y^2)^2}{y} = \displaystyle\lim_{y\to 0} \dfrac{-y^4}{y^4} = -1$.
    And, $f_{yx}(0, 0) = 1$.
    Therefore, $f_{xy}(0, 0) \neq f_{y}(0, 0)$.

    \item Since $Df$ is continuous on $S$(close interval in $R^p$), $Df$ is Riemann intergrable on $S$.
    
    
\end{enumerate}
\end{document} 