\documentclass[12pt]{article}

\topmargin -40pt
\marginparwidth 0pt
 \oddsidemargin  -40pt
 \evensidemargin 0pt
 \marginparsep 0pt
\textwidth 7.2 in
 \textheight  10 in
 \hoffset  0.1in
\linespread{1.9}

\usepackage{amsthm,amsmath,amssymb,amscd,verbatim,epsfig}
\usepackage{mathptmx}
\usepackage{amsfonts}
%\usepackage{setapace}
\usepackage{graphicx}
\usepackage{bm}
%\usepackage{CJK}
\usepackage{ulem}
\usepackage{multicol}
\usepackage{enumerate}
\usepackage{float}
\usepackage{fontspec}
\usepackage{xeCJK}
\setmainfont{Times New Roman}
\setCJKmainfont{TaipeiSansTCBeta-Regular}
\XeTeXlinebreaklocale "zh"
\XeTeXlinebreakskip = 0pt plus 1pt

\title{Homework 8 of Introduction to Analysis(II)}
\author{AM15 黃琦翔 111652028}

\begin{document}
\maketitle
\begin{enumerate}
    \item $f_x(0, 0) = \displaystyle\lim_{x\to 0} \dfrac{f(x, 0) - f(0)}{x} = \displaystyle\lim_{x\to 0} \dfrac{x\cdot 0(x^2-0^2)/(x^2+0^2)}{x} = 0$ exists.
    Also, we can esaily get that $f_y(0, 0) = 0$.
    Then, $f_{xy}(0, 0) = \dfrac{\partial f_x}{\partial y}(0, 0) = \displaystyle\lim_{y\to 0} \dfrac{y(0^4+4\cdot 0^2y^2 - y^4)/(0^2 + y^2)^2}{y} = \displaystyle\lim_{y\to 0} \dfrac{-y^4}{y^4} = -1$.
    And, $f_{yx}(0, 0) = 1$.
    Therefore, $f_{xy}(0, 0) \neq f_{yx}(0, 0)$.

    \item Since $Df$ is continuous on $S$, for any $\epsilon > 0$, there exists $\delta > 0$ s.t. $\| f(x) - f(y)\| < \dfrac{\epsilon}{\|b-a\|}$ if $\| x - y\| < \delta$.
    And since $S$ is a closed line in $R^p$, we can find a sequence $\{ x_k\}_{k=1}^n$ s.t. $D(x_k, \dfrac{\delta}{2}) \supseteq S$ and $\|x_{k+1} - a\| > \| x_k - a\|$.
    Let $x_0 = a$ and $x_{n+1} = b$ and $x_k = a + t_k(b-a)$.
    Then, by MVT, $f(x_k) - f(x_{k-1}) = Df(c_k)(x_k - x_{k-1})$ for $c_k$ on the line between $x_k, x_{k-1}$.
    Thus, 
    \begin{align*}
        |f(b) - f(a) - \int_{0}^{1} Df(a + t(b-a))(b - a)dt| &\leq \sum_{k=1}^{n+1} \|f(x_k) - f(x_{k-1}) - \int_{t_{k-1}}^{t_k} Df(a + t(b-a))(b-a) dt\|\\
        &= \sum_{k=1}^{n+1} \| Df(c_k)(x_k - x_{k-1})\\
        & - \int_{t_{k-1}}^{t_k} Df(a +t(b-a))(b-a)\ dt\|\\
        &= \sum_{k=1}^{n+1} \| \int_{t_{k-1}}^{t_k} Df(c_k)(b-a) - Df(a+t(b-a))(b-a)dt\|\\
        &= \sum_{k=1}^{n+1} \epsilon(t_k - t_{k-1})\\
        &= \epsilon
    \end{align*}

    Therefore, $f(b) - f(a) = \displaystyle\int_{0}^{1} Df(tb + (1-t)a)(b-a)\ dt$.

    \newpage
    \item Since $B$ is bilinear, $g(x + u) = B(x+u, x+u) = B(x, x+u) + B(u, x+u) = B(x, x) + B(x, u) + B(u, x) + B(u, u) = g(x) + g(u) + (B(u, x) + B(x, u))$.
    
    And since $B$ is bilinear, $B(0, 0) = 0$ and $g(0) = 0$.

    For any $x, y\in \mathbb{R}^p$, \begin{align*}
        \| B(x, u) + B(u, x) - Dg(x)(u)\| &= \| cB(x, u') + cB(u', x) - cDg(x)(u')\|\\
        &\leq c(\| g(x) + g(u') + B(x, u') + B(u', x) - Dg(x)(u')\| + \| g(u')\|)\\
        &\leq c(\epsilon\| u'\| + \| g(u')\|)\\
        &\leq 
    \end{align*}
    
    
    \item For any $x_0, y_0 \in \mathbb{R}$, $f(x_0, y_0), f_x(x_0, y_0), f_y(x_0, y_0), f_{xy}(x_0, y_0)$ are all continuous.
    Then, \begin{align*}
        \lim_{h\to 0} \dfrac{f_y(x_0+h, y_0) - f_y(x_0, y_0)}{h} &= \lim_{h, k\to 0} \dfrac{1}{h}\dfrac{f(x_0 + h, y_0 + k) - f(x_0 + h, y_0)}{k} - \dfrac{f(x_0, y_0 + k) - f(x_0, y_0)}{k}\\
        &= \lim_{h, k\to 0} \dfrac{1}{k}\dfrac{f(x_0 + h, y_0+k) - f(x_0, y_0 + k)}{h} - \dfrac{f(x_0+h, y_0) - f(x_0, y_0)}{h}\\
        &= \lim_{k\to 0} \dfrac{f_x(x_0, y_0+k) - f_x(x_0, y_0)}{k}\\
        &= f_{xy}(x_0, y_0)
    \end{align*}

    Thus, $f_{yx}$ exists and $f_{yx} = f_{xy}$.
\end{enumerate}
\end{document} 