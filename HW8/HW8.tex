\documentclass[12pt]{article}

\topmargin -40pt
\marginparwidth 0pt
 \oddsidemargin  -40pt
 \evensidemargin 0pt
 \marginparsep 0pt
\textwidth 7.2 in
 \textheight  10 in
 \hoffset  0.1in
\linespread{1.9}

\usepackage{amsthm,amsmath,amssymb,amscd,verbatim,epsfig}
\usepackage{mathptmx}
\usepackage{amsfonts}
%\usepackage{setapace}
\usepackage{graphicx}
\usepackage{bm}
%\usepackage{CJK}
\usepackage{ulem}
\usepackage{multicol}
\usepackage{enumerate}
\usepackage{float}
\usepackage{fontspec}
\usepackage{xeCJK}
\setmainfont{Times New Roman}
\setCJKmainfont{TaipeiSansTCBeta-Regular}
\XeTeXlinebreaklocale "zh"
\XeTeXlinebreakskip = 0pt plus 1pt

\title{Homework 8 of Introduction to Analysis(II)}
\author{AM15 黃琦翔 111652028}

\begin{document}
\maketitle
\begin{enumerate}
    \item $f_x(0, 0) = \displaystyle\lim_{x\to 0} \dfrac{f(x, 0) - f(0)}{x} = \displaystyle\lim_{x\to 0} \dfrac{x\cdot 0(x^2-0^2)/(x^2+0^2)}{x} = 0$ exists.
    Also, we can esaily get that $f_y(0, 0) = 0$.
    Then, $f_{xy}(0, 0) = \dfrac{\partial f_x}{\partial y}(0, 0) = \displaystyle\lim_{y\to 0} \dfrac{y(0^4+4\cdot 0^2y^2 - y^4)/(0^2 + y^2)^2}{y} = \displaystyle\lim_{y\to 0} \dfrac{-y^4}{y^4} = -1$.
    And, $f_{yx}(0, 0) = 1$.
    Therefore, $f_{xy}(0, 0) \neq f_{y}(0, 0)$.

    \item Since $Df$ is continuous on $S$, for any $\epsilon > 0$, there exists $\delta > 0$ s.t. $\| f(x) - f(y)\| < \dfrac{\epsilon}{\|b-a\|}$ if $\| x - y\| < \delta$.
    And since $S$ is a closed line in $R^p$, we can find a sequence $\{ x_k\}_{k=1}^n$ s.t. $D(x_k, \dfrac{\delta}{2}) \supseteq S$ and $\|x_{k+1} - a\| > \| x_k - a\|$.
    Let $x_0 = a$ and $x_{n+1} = b$ and $x_k = a + t_k(b-a)$.
    Then, by MCT, $Df(x_k) - Df(x_{k-1}) = Df(c_k)(x_k - x_{k-1})$ for $c_k$ on the line between $x_k, x_{k-1}$.
    Thus, 
    \begin{align*}
        |f(b) - f(a) - \int_{0}^{1} Df(a + t(b-a))(b - a)dt| &\leq \sum_{k=1}^{n+1} \|f(x_k) - f(x_{k-1}) - \int_{t_{k-1}}^{t_k} Df(a + t(b-a))(b-a) dt\|\\
        &= \sum_{k=1}^{n+1} \| Df(c_k)(x_k - x_{k-1})\\
        & - \int_{t_{k-1}}^{t_k} Df(a +t(b-a))(b-a)\ dt\|\\
        &= \sum_{k=1}^{n+1} \| \int_{t_{k-1}}^{t_k} Df(c_k)(b-a) - Df(a+t(b-a))(b-a)dt\|\\
        &= \sum_{k=1}^{n+1} \dfrac{\epsilon(t_k - t_{k-1})}{\| (b-a)\|}\\
        &= \epsilon
    \end{align*}

    Therefore, $f(b) = f(a) = \displaystyle\int_{0}^{1} Df(tb + (1-t)a)(b-a)\ dt$.

    \newpage
    \item $\displaystyle\lim_{u \to 0} \dfrac{\|g(x+u) - g(x) - Dg(x)(u)\|}{\| u\|} = \displaystyle\lim_{u\to 0} \dfrac{\|g(x) + g(u) + B(x, u) + B(u, x) - g(x) - Dg(x)(u) \|}{\| u\|} = 0$.
    
    \item For any $(x_0, y_0)\in \mathbb{R}^2$ and tiny $h, k\in \mathbb{R}$,
    since $\dfrac{\partial^2 f}{\partial x\partial y}$ exists and continuous, 
    $\dfrac{\partial f}{\partial x}(x_0, y_0), \dfrac{\partial f}{\partial x}(x_0+h, y_0),$ 
    
    $\dfrac{\partial f}{\partial x}(x_0, y_0+k), \dfrac{\partial f}{\partial x}(x_0+h, y_0+k)$
    are all exists.

\end{enumerate}
\end{document} 