\documentclass[12pt]{article}

\topmargin -40pt
\marginparwidth 0pt
 \oddsidemargin  -40pt
 \evensidemargin 0pt
 \marginparsep 0pt
\textwidth 7.2 in
 \textheight  10 in
 \hoffset  0.1in
\linespread{1.9}

\usepackage{amsthm,amsmath,amssymb,amscd,verbatim,epsfig}
\usepackage{mathptmx}
\usepackage{amsfonts}
%\usepackage{setapace}
\usepackage{graphicx}
\usepackage{bm}
%\usepackage{CJK}
\usepackage{ulem}
\usepackage{multicol}
\usepackage{enumerate}
\usepackage{float}
\usepackage{fontspec}
\usepackage{xeCJK}
\setmainfont{Times New Roman}
\setCJKmainfont{TaipeiSansTCBeta-Regular}
\XeTeXlinebreaklocale "zh"
\XeTeXlinebreakskip = 0pt plus 1pt

\title{Homework 4 of Introduction to Analysis(II)}
\author{AM15 黃琦翔 111652028}

\begin{document}
\maketitle
\begin{enumerate}
    \item Since $B \subset C(A, \mathbb{R}^n)$, $B_x$ is totally bounded.
    From lemma mentioned in class, since $A$ is compact and $B_x$ is totally bounded, $B$ is totally bounded.
    And since $B$ is equicontinuous, for all $\epsilon> 0$, there exists $\delta > 0$ s.t. $d(x, y) < \delta\implies \| f(x) - f(y) < \epsilon\|$ for all $f\in B$ and $x, y \in A$.
    And we can find $\{ f_k\in A\}$ s.t. $B \subseteq \cup D(f_k, \epsilon)$.


    Thus, every sequence in $B$ has uniformly convergent subsequence.

    \item First, we want to show that $B$ is closed.
    For any sequence $f_k \in B$ which converges to $f$, 
    since $f_k(0) = 0$ for all $k$, we can get $f(0) = 0$.
    Then, assume there exists an $x_0, x_1 \in (0, 1)$ s.t. $\dfrac{|f(x_0) - f(x_1)|}{|x_0 - x_1|} = \alpha > 1$.
    Take $\epsilon = \dfrac{\alpha-1}{3}$, there exists $N \in \mathbb{N}$ s.t. $|f(x) - f_k(x)| < \epsilon$ for all $x$ and $k > N$.
    Thus, $|\dfrac{f_k(x_0) - f_k(x_1)}{x_0 - x_1}| = \dfrac{|f_k(x_0) - f(x_0) + f(x_0) - f(x_1) + f(x_1) - f_k(x_1)|}{|x_0 - x_1|} > \alpha - \dfrac{2\epsilon}{|x_0 - x_1|} > 1$ 
    which causes contradiction to $|f_k'(x)| \leq 1$ for all $x\in (0, 1)$.
    Thus, $\dfrac{|f(x_0) - f(x_1)|}{|x_0 - x_1|} \leq 1\implies f\in B$.
    Hence, $B$ is closed.
    
    Then, since $|f'(x)| \leq 1$ for all $f\in B$ and $x\in (0, 1)$.
    For any $\epsilon > 0$, we take $\delta < \epsilon$,
    for any $x, y \in [0, 1]$ s.t. $|x - y| < \epsilon$, 
    $|f(x) - f(y)| \leq 1 \cdot |x-y| = \delta < \epsilon$.
    Thus, $B$ is equicontinuous.

    Last, for any $x \in [0, 1]$, we want to proof $B_x$ is compact.
    For $x = 0$, $B_0 = \{ 0\}$ obviously compact.
    For $x > 0$, since $|f'|\leq 1$, we can easily get $B_x = [-x, x]$ by $f(x) = ax$ for all $|a| \leq 1$.
    Thus, $B$ is pointwise compact.

    Therefore, by Arzela-Ascoli Theorem, $B$ is compact.
    
    \newpage
    \item For any $\epsilon > 0$, there exists $\delta > 0$ s.t. $\| x-y\| < \delta \implies \| f_k(x) - f_k(y)\| < \dfrac{\epsilon}{3}$.
    And since $K$ is compact, we can find finite $p\in N$ and $\{ x_k\}_{k=1}^p$ s.t. $K \subseteq \displaystyle\bigcup_{k=1}^p D(x_k, \delta)$.
    Also, for all $x_k$, we can find a $N_{x_k}\in \mathbb{N}$ s.t. $|f_n(x_k) - f_m(x_k)| < \dfrac{\epsilon}{3}$ for all $n, m > N_{x_k}$.

    Then, we take $N = \displaystyle\max_{x\in [1, p]}\{ N_{x_k}\}$, and for any $x\in K$, we can find a $k\in [1, p]$ s.t. $x \in D(x_k, \delta)$.
    Thus, \begin{align*}
        |f_n(x) - f_m(x)| &\leq |f_n(x) - f_n(x_k)| + |f_n(x_k) - f_m(x_k)| + |f_m(x_k) - f_m(x)|\\
        &< \dfrac{\epsilon}{3} + \dfrac{\epsilon}{3} + \dfrac{\epsilon}{3}\\
        &= \epsilon
    \end{align*}
    Thus, by Cauchy Criterion, $f_k \to f$ uniformly.

    \item Since $\mathfrak{F}$ is equicontinuous, for any $\epsilon > 0$, there exists $\delta > 0$ s.t. $\| x-y\| < \delta \implies |f(x) - f(y)| < \dfrac{\epsilon}{2}$ for all $f\in \mathfrak{F}$.
    For all $x, y\in D$ and $\|x - y\| < \delta$, we want to show $|f^*(x) - f^*(y)| < \epsilon$.
    First, if $f^*(x) > f^*(y)$, we can find a $f\in \mathfrak{F}$ s.t. $f^*(x) - f(x) < \dfrac{\epsilon}{2}$.
    Then, $f^*(y) > f^*(x) - \dfrac{\epsilon}{2} - |f(x) - f(y)| > f^*(x) - \epsilon$.
    For the other case, just change $x, y$ and get the same result.
    Thus, $f^*$ is continuous.
\end{enumerate}
\end{document} 